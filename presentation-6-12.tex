\documentclass[10pt,a4paper]{beamer}
\usepackage[utf8]{inputenc}
\usepackage{amsmath}
\usepackage{braket}
\usepackage[spanish]{babel}
\usepackage{caption}
\usepackage{amsfonts}
\usepackage{amssymb}
\usepackage{graphicx}
\usepackage{hyperref}
\usepackage{slashed}
\usepackage{feynmf}
\usepackage[backend=bibtex, style=authortitle]{biblatex}
\bibliography{bibICTP.bib}

\newcommand{\dLg}{\ensuremath{\mathcal{L}}}
\begin{document}
\newcommand{\Id}{\mathbb{I}} 
\newcommand{\wf}{\Ket{\psi}} 
\newcommand{\num}{\ensuremath{\hat{n}}}
\newcommand{\Lac}{\ensuremath{\mathbf{\hat{L}^2}}} 
\newcommand{\Lag}{\mathcal{L}} 
\newcommand{\Ham}{\ensuremath{\hat{H}}} 
\newcommand{\Tr}{\mathrm{Tr}} 
\title{
  \it{Conformal Bootstrap:} La magia de la simetría.
}
\author{ 
  Luviano Valenzuela Uriel Adrián}
\institute{
  International School for Advanced Studies (SISSA)\\
	Trieste, Italia.}
\date{6 de diciembre de 2018.}
\begin{frame}
\titlepage
\end{frame} 

\begin{frame}{Menú}
\tableofcontents
\end{frame}
\section{Introducción.}
\begin{frame}{\secname}
  \begin{itemize}
    \item Las teorías de campos conformes (CFT) tienen una gran cantidad de
      aplicaciones.

  \end{itemize}
\end{frame}

%%%%%%%%%%%%%%%%%%%%%%%%%%%%%%
\section{Álgebra de SUSY: propiedades y espectro de materia.}
\label{sec:susy-alg}
\subsection{Grupo de Lorentz.}
\begin{frame}{ \subsecname} 
  Recordemos una transformación general de Poincaré:
\[
  x^\mu \to x'^\mu = \Lambda^\mu_\nu x^\nu + a^\mu,\quad 
  \Lambda^T\eta\Lambda = \eta, \quad
\eta^{\mu\nu}=\text{diag}(1,-1,-1,-1).
\]

Su álgebra de Lie está dada por
\begin{equation}
  \label{equ:Lorentz_algebra}
  \begin{aligned}
  \left[P^\mu,P^\nu\right] &= 0\\
    \left[M^{\mu\nu},P^\rho\right] &= i\left(P^\mu\eta^{\nu\rho} -
  P^\nu\eta^{\mu\rho}\right)\\
  \left[M^{\mu\nu},M^{\rho\sigma}\right] &= i\left(M^{\mu\sigma}\eta^{\nu\rho} +M^{\nu\rho}\eta^{\mu\sigma} -
  M^{\mu\rho}\eta^{\nu\sigma} - M^{\nu\sigma}\eta^{\mu\rho} \right).\\
  \end{aligned}
  \end{equation}

  De aquí podemos rescatar una cara más familiar (\emph{boosts} y rotaciones):
\begin{equation}
  \label{equ:lorentz_gens}
  J_i = \frac{1}{2}\epsilon_{ijk}M^{jk} \quad \text{ and }\quad K_i = M_{0i}.
  \end{equation}

  
\end{frame}

%%%%%%%%%%%%%
\begin{frame}{Representaciones.}
  Localmente,
  \[
  A_i = \frac{1}{2}\left(J_i + iK_i\right) \quad \text{y} \quad  B_i = \frac{1}{2}\left(J_i
      - iK_i\right)
  \]
  arman dos copias del álgebra de  $SU(2)$:
      \[
        \left[A_i,A_j\right]= i\sum_k \epsilon_{ijk}A_k,\;
\left[B_i,B_j\right] =i\sum_k \epsilon_{ijk}B_k,\;
\left[A_i,B_j\right] = 0,
\]
es decir,
\[
  SO(1,3)\sim SU(2)\oplus SU(2).
\]
Así, se antoja etiquetar los objetos que transforman bajo Lorentz como $(A,B)$,  $A,B \in \mathbb{Z}
\cup \mathbb{Z} + (1/2)$. 

\end{frame}

\subsection{Representaciones importantes (útiles) de  $SL(2,\mathbb{C})$.}
\begin{frame}{\subsecname}
  \begin{itemize}
 \item  $SO(3,1) \sim SL(2,\mathbb{C})/\mathbb{Z}_2$.
  
\item \emph{Espinores Izquierdos de Weyl}:
as
\[
  \psi_\alpha \to \psi'_\alpha = N_\alpha^\beta \psi_\beta;\quad \alpha,\beta =
  1,2.
\]

\item \emph{Espinores Derechos de Weyl}:
\[
  \psi_{\dot\alpha} \to \psi'_{\dot\alpha} = N_{\dot\alpha}^{* \,\dot\beta}
  \psi_{\dot\beta}; \quad\dot\alpha,\dot\beta = 1,2.
\]
\item Representaciones contravariantes:
\[
  \psi^\alpha = \epsilon^{\alpha\beta} \psi_\beta \quad \text{y} \quad
  \psi^{\dot\alpha} = \epsilon^{\dot\alpha\dot\beta} \psi_{\dot\beta}
\]

\item \emph{Ojo}: $V^\mu \in (1/2,1/2)$, $\psi^\alpha \in (1/2,0)$
  $\Rightarrow$ Los espinores son más fundamentales (maomenos).
  \end{itemize}
\end{frame}
%%%

\subsection{ $SL(2,\mathbb{C})$ y sus generadores.}
\begin{frame}{\subsecname}
\[
  \left( \sigma^{\mu\nu}\right)_\alpha^{\beta} = \frac{i}{2}
  \sigma^{[\mu}
\bar\sigma^{\nu]},\;
%
\left( \bar\sigma^{\mu\nu}\right)^{\dot\alpha}_{\dot\beta} = \frac{i}{2}
  \sigma^{[\mu}
  \bar\sigma^{\nu]},\;\sigma^\mu := (1,\sigma^i)
\]
\[
  \left[\sigma^{\mu\nu},\sigma^{\rho\tau}\right] = i\left(\sigma^{\mu\tau}\eta^{\nu\rho} +\sigma^{\nu\rho}\eta^{\mu\tau} -
  \sigma^{\mu\rho}\eta^{\nu\tau} - \sigma^{\nu\tau}\eta^{\mu\rho} \right).\]
  

  Para los espinores
\[
  J^{(1/2,0)}_i = \frac{1}{2} \epsilon_{ijk} \sigma^{jk} = \epsilon_{ijk}
  \frac{i}{4}\sigma^{j}\bar\sigma^k = -\epsilon_{ijk}
  \frac{i}{4}\sigma^{j}\sigma^k = \frac{1}{2} \sigma _i = J^{(0,1/2)}_i,
\]
pero
\[
  K^{(1/2,0)}_i = \sigma_{0i} = \frac{i}{4}\left(\sigma_{0}\bar\sigma_i -
    \sigma_{i}\bar\sigma_0\right ) = -\frac{i}{4}\sigma_i = -\bar\sigma_{0i} =
    -K^{(0,1/2)}_i.
\]

\end{frame}

\begin{frame}{\emph{Interludio:} Convenciones y notación.}
\[
  \chi\psi := \chi^\alpha\psi_\alpha =
  \epsilon^{\alpha\beta}\chi_\beta\psi_\alpha =
  -\epsilon^{\beta\alpha}\chi_\beta\psi_\alpha = -
  \chi_\alpha\psi^\alpha; 
  %%%%%%%%%%%%%
  \quad \bar\chi\bar\psi =
  \bar\chi_{\dot\alpha}\bar\psi^{\dot\alpha} =
  -\bar\chi^{\dot\alpha}\bar\psi_{\dot\alpha}.
   \]

   Ingenuamente, parecería que
\[
  \psi\psi = \epsilon^{\alpha\beta} \psi_\beta\psi_\alpha = \psi_2\psi_1 -
  \psi_1\psi_2 = 0,
\]

pero el pedir que los componentes sean números anticonmutantes (de Grassmann) nos
da cosas más sensatas:


\[
  \psi\psi = 2 \psi_2\psi_1.
\]

 Bajo conjugación
\[
  \psi_\alpha^\dag = \bar\psi_{\dot\alpha} \Rightarrow \psi^{\alpha \dag} = -
  \bar\psi^{\dot\alpha}.
\]


Expansión de productos:
\begin{equation}
  \psi_\alpha\bar\chi_{\dot\alpha} = \frac{1}{2}\left(\psi\sigma_\mu\bar\chi
  \right)\sigma^\mu_{\alpha\dot\alpha}
  %%%
  \quad \text{y} \quad
  %%
  \bar\psi^{\dot\alpha}\chi^{\alpha} = \frac{1}{2}\left(\bar\psi\bar\sigma_\mu\chi
  \right)\bar\sigma^{\mu\dot\alpha\alpha} .
  \label{eq:sigma}
\end{equation}
\[
  \psi_\alpha\chi_\beta
  =
  \frac{1}{2} \epsilon_{\alpha\beta} \psi\chi
  +
  \frac{1}{2} \left(\psi\sigma_{\mu\nu} \chi\right)
  \left(\sigma^{\mu\nu}\epsilon^T\right)_{\alpha\beta},\quad \quad
  \left(\sigma^{\mu\nu}\epsilon^T\right)_{\alpha\beta} :=
  \left(\sigma^{\mu\nu}\right)_\alpha^{\;\gamma} \epsilon_{\gamma\beta}
\]
\end{frame}

\subsection{SUSY y su Álgebra.}
\begin{frame}{\subsecname}

En 1967, Coleman y Mandula probaron que la simetría bosónica más general que puede tener
la matriz S (de diSpersión) de cualquier teoría física relativamente sensata era
Poincaré $\otimes$ Simetrías Internas. Buscando sacarle la vuelta a este teorema
se propuso una simetría fermiónica. Debido a su espín semientero, los
generadores de esta simetría cambian la estadística de los estados en los que
actúan, por lo que abren la puerta a nuevas y emocionantes posibilidades,
algunas de las cuales discutiremos a continuación.
\end{frame}

\begin{frame}{Álgebras de Lie gradadas.}
  \begin{itemize}
    \item En física, los operadores fermiónicos satisfacen relaciones de
      conmutación, así que debemos introducir el concepto de un álgebra gradada:
\[
  O_aO_b - (-1)^{\eta_a\eta_b} O_bO_a = iC^e_{ab} O_e
\]
\item $\eta =0$  para bosones.
\item $\eta = 1$ para fermiones.
\end{itemize}
\end{frame}

\begin{frame}{Álgebra de SUSY.}
  \begin{itemize}
    \item Generadores: $Q^A_\alpha,\bar
Q^A_{\dot\alpha}$, $A=1,\cdots\mathcal{N}$ 
\item Álgebra extendida (SuperPoincaré):
  \[
  \left[Q^A_\alpha, M^{\mu\nu} \right] = 
\left(\sigma^{\mu\nu} \right)
  _\alpha^\beta  Q^A_\beta, \quad
\left[Q^A_\alpha, P^{\mu} \right] =0, \quad
\{Q^A_\alpha, Q^B_\beta \}
  =
\epsilon_{\alpha\beta} Z^{AB}
\]
\[\{Q^A_\alpha, Q^B_{\dot\beta} \} = 2 \delta^{AB}\sigma^\mu_{\alpha\dot\beta}P^\mu\]

\item Simetrías internas ($T_i$): $[T_i,Q] =0$.
\item Simetría R: $Q'^A = U^A_BQ^B$. En particular, para SUSY sencilla:
\[
[Q,R] = Q, \quad[\bar Q,R] = - \bar Q.
\]
  \end{itemize}
\end{frame}

%%%%%%%%%%%%%%%%%%%%%
\subsection{Conociendo al Zoológico.}
\begin{frame}{Partículas no-masivas.}
\begin{itemize}
  \item Si $m=0$, tomamos $p_0 = p_3 = E$ y $p_2 = p_1 = 0$:
\[\{Q^A_\alpha, \bar Q^B_{\dot\beta} \}
  =
  2 \delta^{AB}2E(\sigma^0+\sigma^3)=4 E\delta^{AB}
  \begin{pmatrix}
    1&0\\
    0&0
  \end{pmatrix}_{\alpha\dot\beta}.
\]

\item Definimos
\[
a^A:= \frac{Q_1^A}{2\sqrt{E}},\quad
  a^{A\dag}:= \frac{\bar Q_{\dot1}^A}{2\sqrt{E}}
\]
y vemos que
$\left[a,J^3\right] = a/2$ and $\left[a^\dag,J^3\right] = -a/2$ 
\item $a^A$ disminuye la helicidad de la partícula en $1/2$ mientras que
$a^{A\dag}$ la aumenta.

\item $\mathcal N =1$: Empecemos con $\ket{\lambda}$ tal que
$a^\dag\ket{\lambda} = 0$. Su supercompañero es:
$a\ket{\lambda}:=\ket{\lambda-1/2}$. 

\item Si las energías que estamos considerando son mayores a la del rompimiento
  electrodébil ($10^{11}$eV), una partícula con helicidad $\lambda$ tiene una
  supercompañera de helicidad $\lambda -1/2$.
\end{itemize}
\end{frame}
%%%%%%

\begin{frame}{SUSY extendida.}

Si la helicidad máxima es $\lambda$, tenemos a los siguientes amiguitos en el
supermultiplete:
\begin{center}
  \begin{tabular}{c|c|c}
    Estado & Helicidad & No. de estados distintos \\
    \hline
  $\ket{\lambda}$ & $\lambda$ & $\begin{pmatrix} 0\\ \mathcal{N}\end{pmatrix}$\\
  $a^{A_1} \ket{\lambda}$ & $\lambda -\frac{1}{2}$ & $\begin{pmatrix} 1\\ \mathcal{N}\end{pmatrix}$\\
  $a^{A_1}a^{A_2} \ket {\lambda}$ & $\lambda-\frac{2}{2}$ & $\begin{pmatrix} 2\\ \mathcal{N}\end{pmatrix}$\\
  $a^{A_1}a^{A_2} a^{A_3}\ket {\lambda}$ & $\lambda-\frac{3}{2}$ & $\begin{pmatrix} 3\\ \mathcal{N}\end{pmatrix}$\\
    $\vdots $&$\vdots$&$\vdots$\\
  $a^{A_1}\cdots a^{A_\mathcal{N}}\ket {\lambda}$ &
  $\lambda-\frac{\mathcal{N}}{2}$ & $\begin{pmatrix} \mathcal{N}\\
  \mathcal{N}\end{pmatrix}$.\\
\end{tabular}
\end{center}
Si requerimos $|\lambda|\leq 2$ entonces
$\mathcal{N}\leq 8$.

\end{frame}
%%%%%%%%%%%%%%%%%%%%MASSIVEEEEEEEEEEEEEE%%%%%%%%%%%%%%%%%%%
\begin{frame}{Estados masivos.}
Para $m\neq 0$ tomamos $p^0=m$ y $p^i=0$
\[\{Q^A_\alpha, Q^B_{\dot\beta} \}
  =
  2 \delta^{AB}m(\sigma^0)=2 m\delta^{AB}
  \begin{pmatrix}
    1&0\\
    0&1
  \end{pmatrix}_{\alpha\dot\beta}.
\]

SUSY sencilla:
\[
a_\alpha:= \frac{Q_\alpha}{\sqrt{2m}},\quad
a^{\dag}_{\dot\alpha}:= \frac{\bar Q_{\dot\alpha}}{\sqrt{2m}}
\]

Si arrancamos con espín$-j\neq 0$ $\ket{\Omega}$ aniquilado por ambos $a_\alpha$
obtenemos
\[a^\dag_1 \ket{\Omega} = k_1\ket{j+1/2,j_3 + 1/2}+ k_2\ket{j-1/2,j_3 +
  1/2}
\]\[
a^\dag_2 \ket{\Omega} = k_3\ket{j+1/2,j_3 - 1/2}+ k_4\ket{j-1/2,j_3 -
  1/2}
  \]
  \[a^\dag_1a^\dag_2 \ket{\Omega} = -a^\dag_2a^\dag_1 \ket{\Omega} \propto
  \ket{\Omega}. \]

Para $j=0$:
\[
  a^\dag_{1,2} \ket{\Omega} = \ket{1/2, \pm1/2}.
\]
\[
  a^\dag_{1}a^\dag_2 \ket{\Omega} =- a^\dag_2 a^\dag_{1}\ket{\Omega} \sim
  \ket{\Omega}.
\]

Definimos estados de paridad conservada
\[
  \ket{+} =\ket{\Omega} +  a^\dag_{1}a^\dag_2 \ket{\Omega}, \quad
  \ket{-} =\ket{\Omega} -  a^\dag_{1}a^\dag_2 \ket{\Omega}.
\]
\end{frame}

%%%%%%%%%%%%%%%%%%%%%%%%%%%%%%%
\begin{frame}{SUSY extendida y estados
  BPS\footnote{Bogomol'nyi-Prasad-Sommerfield.}.}
  
  \begin{itemize}
    \item Ninguna $Q^A$ se muere, así que $\Rightarrow Z^{AB} \neq 0$.
    \item Tenemos que
\[
  m\geq
  \frac{1}{2\mathcal{N}} \Tr\{\sqrt{Z^\dag Z}\}.
\]
\item Los estados que saturan la desigualdad son conocidos como estados BPS y se
  comportan casi casi como si no tuvieran masa ($2^{\mathcal{N}}$ estados). \footcite{wein-susy}

\item Si tomamos $Z^{AB} = 0$, todas las $Q$s juegan, así que tenemos
  $2^{2\mathcal{N}}$ estados.

\end{itemize}

\end{frame}
%%%%%%%%%%%%%%%%%%%%%%%%%%%%%%%%%%%%%%%%%%%%%%%%%%%%%%%%%%%%%%%%%%%%%%%%%%%%%%%
\section{Superespacio y Supercampos.}

\subsection{Superespacio.}
\begin{frame}{\subsecname}
  \[
    \text{Superespacio} \sim \frac{\text{SuperPoincaré}}{\text{Lorentz}}
  \]
  Es decir,
  \[\{a^\mu ,
\theta^\alpha, \bar\theta_{\dot\alpha} \}
\sim
    \{\omega^{\mu\nu},a^\mu ,
\theta^\alpha, \bar\theta_{\dot\alpha} \} /\{\omega^{\mu\nu}\}  .
\]

\end{frame}

%%%%%%%%%
\begin{frame}{Supercálculo.}
  \begin{itemize}
    \item Sean $\theta_\alpha$ y $\bar\theta^{\dot\alpha}$, espinores de Weyl
      con componentes anticonmutativos. Así,
\[
  \theta\theta := \theta^\alpha\theta _\alpha,\quad
  \bar\theta\bar\theta := \bar\theta_{\dot\alpha}\bar\theta ^{\dot\alpha}.
\]
\[
  \theta^\alpha\theta^\beta = -\frac{1}{2}\epsilon^{\alpha\beta} \theta\theta,\quad
  \bar\theta^{\dot\alpha} \bar\theta^{\dot\beta} =
  \frac{1}{2}\epsilon^{\dot\alpha\dot\beta} \bar\theta\bar\theta.
\]
\item Diferenciación:
\[
\frac{\partial\theta^\alpha}{\partial\theta^\beta} = - \frac{\partial\theta_\alpha}{\partial\theta_\beta}
=
\delta^\alpha_\beta
,\quad \text{y} \quad
\frac{\partial\bar\theta^{\dot\alpha}}{\partial\bar\theta^{\dot\beta}} = -\frac{\partial\bar\theta_{\dot\alpha}}{\partial\bar\theta_{\dot\beta}} =
\delta^{\dot\alpha}_{\dot\beta}.
\]

\item Integración:
\[
  \int d^2\theta \theta\theta = 1 \Rightarrow d^2\theta := -\frac{1}{2}\int
  d\theta_1 \int d\theta_2\quad \text{y} \quad \int d^2\theta d^2\bar\theta
  \left(\theta\theta \bar\theta\bar\theta\right) = 1.
\]
\end{itemize}
\end{frame}
%%%%%%%%%%
\subsection{Supercampo escalar.}
\begin{frame}{\subsecname}
  \vspace{-1em}
\[
  S(x^\mu, \theta_\alpha, \bar\theta_{\dot\alpha})
  =
  \phi(x) + \theta\psi(x) + \bar\theta\bar\chi(x) + \theta\theta M(x) +
  \bar\theta\bar\theta N(x) + (\theta \sigma^\mu\bar\theta ) V_\mu(x) 
\]
\[
 + (\bar\theta\bar\theta) \theta\rho(x) +  (\theta\theta)
  \bar\theta\bar\lambda(x) + \theta\theta\bar\theta\bar\theta D(x).
\]

Tenemos 16 GdL fermiónicos y otros 16 bosónicos que se mezclan bajo SUSY así:
\[
  \begin{aligned}
    \delta\phi &= \epsilon\psi +\bar\epsilon\bar\chi 
     \\
    %
    \delta\psi &=  2\epsilon M+\sigma^\mu\bar\epsilon V_\mu+ i\sigma^\mu\bar\epsilon \partial_\mu\phi 
 \\
    %
    \delta\bar\chi &= 
    2\bar\epsilon N
    -i\epsilon\sigma^\mu \partial_\mu\phi +\epsilon\sigma^\mu V_\mu
     \\
    %
    \delta V_\mu
    &=\frac{i}{2}\left(\partial^\nu\psi \sigma_\mu \bar\sigma_\nu \epsilon 
    -  \bar\epsilon\bar\sigma^\nu  \sigma_\mu\partial^\nu\bar\chi\right) 
    +
    \epsilon\sigma_\mu\bar\lambda+ \rho\sigma_\mu\bar\epsilon
\\
    %
    \delta M& = 
    \bar\epsilon\bar\lambda
     -\frac{ i}{2} \partial_\mu\psi\sigma^\mu \bar\epsilon
    \\
    %
    \delta N  &= + \frac{i}{2}\epsilon \sigma^\mu \partial^\mu\bar\chi   
+  \epsilon\rho
    \\
    %
    \delta\bar\lambda & = 2\bar\epsilon D   
      - i \theta\sigma^\mu\bar\theta \theta\sigma^\nu\bar\epsilon \partial_\nu V_\mu
      +i(\epsilon\sigma^\mu)\partial_\mu M
    \\
    %
    \delta \rho  &=   2\epsilon D - i \theta\sigma^\mu\bar\theta \epsilon\sigma^\nu\bar\theta \partial_\nu V_\mu
    +i(\sigma^\mu\bar\epsilon)
    \partial_\mu N     \\
    %
    \delta D &=\frac{i}{2} \partial_\mu \left( \epsilon\sigma^\mu\bar\lambda -
    \rho\sigma^\mu \bar\epsilon\right).
  \end{aligned}
\]

\end{frame}

%%%%%%%%%%%%%%%%%55
\begin{frame}{En busca de supercampos más razonables.}
\begin{itemize}
  \item 16 GdL son muchos :(.
  \item Definimos una derivada supercovariante para formular restricciones
    compatibles con SUSY:
\[
  \mathcal{D}_\alpha : =  \partial_\alpha + i\sigma^\mu_{\alpha\dot\beta}
  \bar\theta^{\dot\beta} \partial_\mu,\quad
  %%%%
  \mathcal{\bar D}_{\dot\alpha} : =  -\partial_{\dot\alpha} - i\theta^{\beta}\sigma^\mu_{\alpha\dot\beta}
   \partial_\mu,
\]
satisface
\[
  \left\{\mathcal{D}_\alpha, \mathcal{Q}_\beta \right\} =0,\quad
  \left\{\mathcal{D}_\alpha, \mathcal{D}_\beta \right\} =  \left\{\mathcal{\bar
  D}_{\dot\alpha}, \mathcal{\bar D}_{\dot\beta } \right\} =0,\quad  
\]
y
\[\left\{\mathcal{D}_\alpha, \mathcal{\bar D}_{\dot\beta }\right\} =
  -2i\sigma^\mu_{\alpha\dot\beta} \partial_\mu.
\]


\end{itemize}
\end{frame}
\subsection{Supercampo Quiral.}
\begin{frame}
Si pedimos $\mathcal{\bar D}_{\dot\alpha} \Phi = 0$ obtenemos un supercampo
quiral,
\[
  \Phi = \phi(y) + \sqrt{2} \theta \psi (y) + \theta \theta F(y); \quad y: = x^\mu + i \theta\sigma^\mu\bar\theta.
\]
Y obtenemos $4+4$ GdL que transforman como
\[
  \begin{aligned}
    \delta \phi &= \sqrt{2}\epsilon \psi\\
    \delta \psi &= \sqrt{2}i\sigma^\mu\bar\epsilon \partial_\mu\phi +
    \sqrt{2}\epsilon F\\
    \delta F    &= \sqrt{2}i\bar\epsilon\bar\sigma^\mu\partial_\mu \psi.
  \end{aligned}
\]


\end{frame}

%%%%%%%%%%%%%%%%%%%%
\subsection{Supercampo Vectorial.}
\begin{frame}{\subsecname}
Imponiendo $V^\dag = V$ obtenemos un supercampo real y con una especie de
invarianza de norma:
\[
  V(x,\theta,\bar\theta) = 
  (\theta\sigma^\mu\bar\theta) V_\mu(x) +
  (\theta\theta)(\bar\theta\bar\lambda(x)) + 
  (\bar\theta\bar\theta)(\theta\lambda(x)) +
  \frac{1}{2}(\theta\theta\bar\theta\bar\theta) D(x).
\]
En este supermultiplete hay un vector (2 GdL) un campo auxiliar complejo (2 GdL)
y un espinor complejo (4GdL). Estos grados de libertad se mezclan así bajo
supersimetría:
\[
  \begin{aligned}
    \delta V_\mu &= \epsilon \sigma_\mu \bar\lambda +
    \lambda\sigma_\mu\bar\epsilon\\
    %d
    \delta \lambda &= {2}\epsilon D+
    \frac{i}{2}\bar\sigma^\mu\sigma^\nu\bar\epsilon \partial_\mu V_\nu\\
    \delta D    &= \frac{i}{2}\partial_\mu\left(\epsilon\sigma^\mu\bar\lambda
    -\lambda\sigma^\mu\bar\epsilon
  \right).
  \end{aligned}
\]
\end{frame}

\subsection{Construyendo un Lagrangiano supersimétrico.}
\begin{frame}{\subsecname}
  Para que nuestra teoría sea invariante bajo SUSY, queremos
$\delta_\epsilon \Lag = \partial_\mu j^\mu$.
$F$ y $D$ transforman como derivadas totales, así que proponemos
\[
  \Lag
  =
  K(\Phi,\Phi^\dag)\Big|_D + \left(W(\Phi)\Big|_F + h.c.\right),
\]
donde $K$ es el potencial de Kähler, una función real (simétrica en sus argumentos) y $W$ es una función
holomorfa conocida como superpotencial.

\end{frame}

\section{Rompimiento de SUSY.}
\begin{frame}{\secname}
  \label{sec:breaking}
  Evidentemente, nuestro universo no es supersimétrico (en el rango de energías
  que hemos explorado), por lo que desde un punto de vista fenomenológico, es
  indispensable proponer un mecanismo de rompimiento de SUSY.

Pero, ¿\emph{cómo}?

\end{frame}

\subsection{Rompimiento espontáneo de SUSY.}
\begin{frame}{\subsecname}
Cuando el vacío de una teoría no comparte todas las simetrías del sistema
completo, se dice que la simetría simetría se rompió espontáneamente. Esto
equivale a pedirle al vacío que no transforme trivialmente bajo la acción de los
generadores de la simetría en cuestión. En nuestro caso, queremos que $Q\ket{\Omega}\neq 0$.

En la siguiente sección veremos cómo se logra esto.
\begin{center}
\includegraphics[width=0.6\linewidth]{higgs.png}
\end{center}
\end{frame}

\begin{frame}{Rompimiento a partir del término $F$.}
Si tenemos un supercampo quiral $\Phi$ con componentes $(\phi,\psi,F)$.
\[
  \begin{aligned}
    \delta \phi &= \sqrt{2}\epsilon \psi\\
    \delta \psi &= \sqrt{2}i\sigma^\mu\bar\epsilon \partial_\mu\phi +
    \sqrt{2}\epsilon F\\
    \delta F    &= \sqrt{2}i\bar\epsilon\bar\sigma^\mu\partial_\mu \psi.
  \end{aligned}
\]
Para que el campo no transforme trivialmente, es necesario que
$\braket{\psi}\neq 0$, $\braket{\partial_\mu \phi}\neq 0$ o $\braket{F}\neq 0$.

La única opción invariante de Lorentz es $\braket{F}\neq 0$.
\end{frame}

\begin{frame}{Rompimiento a partir del término $D$.}
$V$ supercampo vectorial con componentes $(V_\mu,\lambda,D)$
Sus componentes se mezclan así:
\[
  \begin{aligned}
    \delta V_\mu &= \epsilon \sigma_\mu \bar\lambda +
    \lambda\sigma_\mu\bar\epsilon\\
    %d
    \delta \lambda &= {2}\epsilon D+
    \frac{i}{2}\bar\sigma^\mu\sigma^\nu\bar\epsilon \partial_\mu V_\nu\\
    \delta D    &= \frac{i}{2}\partial_\mu\left(\epsilon\sigma^\mu\bar\lambda -\lambda\sigma^\mu\bar\epsilon
  \right).
  \end{aligned}
\]
De nuevo, la única opción sensata es
$\braket{D}\neq 0$.
\end{frame}




\subsection{Rompimiento suave de SUSY.}
\begin{frame}{\subsecname}
\begin{itemize}
  \item Términos de la forma $\lambda_f H\bar f f$ o $\lambda_S H \bar H S \bar S$
    introducen divergencias cuadráticas al calcular correcciones radiativas de
    la masa del Higgs (Problema de la Jerarquía Electrodébil).
  \item Una de las principales ventajas de SUSY es la cancelación ``milagrosa''
    de estas contribuciones.
  \item Si se propone un lagrangiano con términos explícitamente no
    supersimétricos, es necesario que sus acolpamientos sean de dimensión de masa positiva
    para evitar echar a perder el asunto.
  \item Los mentados \emph{términos suaves} son de la forma
\end{itemize}
\[
  \Lag_{soft} =
  -\left( \frac{1}{2} M_a \lambda^a\lambda^a +
    \frac{1}{6}a^{ijk}\phi_i\phi_j\phi_k
    + \frac{1}{2}b^{ij}\phi_i\phi_j
    + t^{i}\phi_i
  \right) + h.c.
  - \frac{1}{2}(m^2)^{ij}\phi_i^*\phi_j.
\]

\end{frame}
%%%%%%%%%%%%%%%%%%%% %%%%%%%%%%%%%%%%%%%% %%%%%%%%%%%%%%%%%%%% %%%%%%%%%%%%%%%%%%%% %%%%%%%%%%%%%%%%%%%%
%%%%%%%%%%%%%%%%%%%%
%%%%%%%%%%%%%%%%%%%%%%%%%%%%%%%%%%%%%%%%%%DEATH AND DESTRUCTION%%%%%%%%%%%%%%%%% %%%%%%%%%%%%%%%%%%%%
%%%%%%%%%%%%%%%%%%%%
%%%%%%%%%%%%%%%%%%%% %%%%%%%%%%%%%%%%%%%% %%%%%%%%%%%%%%%%%%%% %%%%%%%%%%%%%%%%%%%% %%%%%%%%%%%%%%%%%%%%
\section{KKLT+ correcciones $\alpha'$: El método del Goldstino Nilpotente y su
comparación con resultados previos.}
\label{sec:KKLT}
%%%%%%%%%%%%%%%
\begin{frame}{Teoría de Cuerdas tipo IIB.}
\begin{itemize}
  \item Cuerdas supersimétricas orientadas. Teoría quiral ($\mathcal{N} =
    (2,0)$).
  \item Para ser consistente debe de vivir en 10 dimensiones espaciotemporales.
  \item Hay que compactificar las seis dimensiones restantes para obtener
    fenomenologías razonables (más o menos).
  \item Todos los parámetros libres (módulos) de estas variedades se comportan
    como campos y deben de ser estabilizados para que el modelo no prediga
    salvajadas.
  \item Un factor común de las teorías de cuerdas tipo II es que su límite a
    bajas energías es algún tipo de supergravedad.
\end{itemize}

\end{frame}
%%%%%%%%%%%%%%%%%%%%%%%5
\begin{frame}{KKLT: un repasón.}
\begin{itemize}
  \item En el 2003, Kachru, Kallosh, Linde and Trivedi (KKLT) lograron
    estabilizar todos los módulos de una teoría de cuerdas tipo IIB
    \footcite{KKLT}. Además, por
    medio de una  $\bar D3-$ brana lograron elevar la energía del vacío a un
    valor positivo  --como las observaciones cosmológicas lo requieren-- y
    romper supersimetría. 
  \item En el 2005, Choi et al \footcite{Nilles} lograron calcular los términos
    suaves de este modelo.
  \item En 2015, Aparicio et al. \footcite{peiper} hicieron lo mismo pero usando
  un campo nilpotente ($XX=0$) para representar el efecto de la $\bar D -3$brana
y considerando correcciones de orden superior provenientes de la teoría de
cuerdas.
\item Es interesante ver si los resultados de Choi et al. pueden ser extendidos
  a órdenes superiores en $\alpha'$ y ver si éstos coinciden con los de Aparicio
  et al.
\end{itemize}
\end{frame}

\subsection{KKLT con goldstino nilpotente ($XX=0$).}\label{sec:nilpotent}
\begin{frame}{\subsecname}
  Potencial de Kähler:
%
\begin{equation}
  \label{ka2}
  \begin{aligned}
    K &=-\text{Log}[2 s]-2 \text{Log}\left[\tau^{3/2}-\frac{1}{2} s^{3/2} \xi\right] 
        +\frac{ \phi \bar \phi \alpha _0 }{\tau }\left(1-\frac{s^{3/2} \xi \alpha _1}{\tau ^{3/2}}\right)\\
  &+\frac{ X \bar X\beta _0 }{\tau }\left(1-\frac{s^{3/2} \xi  \beta _1}{\tau
  ^{3/2}}\right)+
  \frac{X \bar X  \phi \bar \phi \gamma _0
  }{\tau ^2}\left(1-\frac{s^{3/2} \xi  \gamma _1}{\tau ^{3/2}}\right),
\end{aligned}
\end{equation}
donde $ T = \tau + i \psi$ y $S=s+ic$. El volumen de la variedad compacta está
dado por
$\mathcal{V} \sim \tau^{3/2}$ y $S$ parametriza la intensidad de las
interacciones perturbativas entre cuerdas.

Superpotencial:
\[W = W_0+A e^{-a T}+X \rho\,.\]
\end{frame}

\begin{frame}{Potencial Escalar.}

  En SUGRA, el potencial escalar se calcula como
  \[
    V =
  e^K\{
    K_{i\bar j}^{-1}D_iWD_{\bar j} W^* - 3|W|^2\}, \; 
    K_{i \bar j} = \partial_i \partial_{ \bar j} K, \; 
    D_i = \partial_i + \partial_i K.
\]


Así, a primer orden en  $\xi$ y tomando $\phi =0$, tenemos
que la contribución de la $\bar D3-$brana es 

\begin{equation}
   \label{uplift}
     V_{\alpha'up}=\frac{\rho ^2}{2 s \beta_0 \mathcal{V} ^{4/3}}\left[1+(1+\beta_1) \frac{ \xi
     \left(s^{3/2} \right)}{ \mathcal{V}}\right] \,.
 \end{equation}

 el resto del potencial escalar está dado por
\begin{equation}
  \label{scalar}
  \begin{aligned}
    V _{\alpha'KKLT}&= 
       \frac{2 a A^2 e^{-2 a \tau}}{s \mathcal{V}^{4/3}}+
       \frac{2 a^2 A^2 e^{-2 a \tau}}{3 s \mathcal{V}^{2/3}}-
       \frac{2 a A e^{-a \tau} W_0}{s \mathcal{V}^{4/3}}\\
       &-
       \frac{3 A^2 e^{-2 a \tau} \sqrt{s} \xi }{8 \mathcal{V}^3}+
       \frac{a A^2 e^{-2 a \tau} \sqrt{s} \xi }{2 \mathcal{V}^{7/3}}\\
       &-\frac{a^2 A^2 e^{-2 a \tau} \sqrt{s} \xi }{6 \mathcal{V}^{5/3}}+
       \frac{3 A e^{-a \tau} \sqrt{s} W_0 \xi }{4 \mathcal{V}^3}\\
       &-
       \frac{a A e^{-a \tau} \sqrt{s} W_0 \xi }{2 \mathcal{V}^{7/3}}-
       \frac{3 \sqrt{s} W_0^2 \xi }{8 \mathcal{V}^3}.
  \end{aligned}
\end{equation}
\end{frame}

%%%%%
\begin{frame}{Minimizando el potencial escalar}
  Para facilitar la tarea, podemos minimizar el potencial sin correcciones ($\xi
  \to 0$). Así, encontramos que $\tau$ debe satisfacer
  \begin{equation}
  \label{min}
    W _0= 
    \frac{1}{3} A e^{-a \tau} (3+ 2a\tau).
  \end{equation}

  $\Rightarrow$ Para valores sensatos de $A$, $a$ y $W_0$, $\tau$ se estabiliza.

\end{frame}

%%%%

\begin{frame}{Masas escalares}

  Expandiendo $V$ a primer orden en 
$|\hat\phi|^2$, donde $\hat \phi = \sqrt{K_{\phi\bar\phi}}\phi$, se obtienen las
masas escalares. Así,
  \begin{equation}
  \label{v2}
  V_2 = V_1 + \left[ \frac{2}{3}(V_{\alpha'KKLT} +
    V_{\alpha'up} ) + \Theta _{\text{up} } + \Theta_{\text{up}\alpha'} +    \tilde \Theta_{\alpha'}\right] |\hat\phi|^2,
\end{equation}
donde
  \begin{equation}
  \label{masas}
  \begin{aligned}
    \Theta_{up} &= \frac{\rho ^2}{2 s \tau ^2 \beta _0}\left(1 +
                   \frac{s\sqrt{s} \xi }{\mathcal{V} }\left(1+\beta_1\right)\right)\left(\frac{1}{3}-\frac{\gamma_0}{\alpha _0 \beta _0}\right) 
                 = \frac{1}{3}V_{\alpha'up}\left(1-\frac{3\gamma_0}{\alpha _0 \beta _0}\right)\\
%%%
    \Theta_{\alpha'up} &= \frac{\rho ^2 \sqrt{s} \xi }{2 \mathcal{V} ^{7/3} \beta _0}\frac{\gamma _0 }{ \alpha _0 \beta_0}
                                \left(\gamma _1- \alpha _1-\beta_1\right)\\
%%%
    \tilde \Theta _{\alpha'}&=\frac{5\sqrt{s} \xi  }{2}
     \bigg[ \frac{ a^2 A^2 e^{-2 a \tau} }{9 \mathcal{V}^{5/3}}
           -\frac{ a A e^{-a \tau} W_0  }{3 \mathcal{V}^{7/3}}
           +\frac{  W_0^2  }{4 \mathcal{V}^3} \\
       & \hskip 1.5cm
           +\frac{ A^2 e^{-2 a \tau}  }{4 \mathcal{V}^3}
           +\frac{ a A^2 e^{-2 a \tau}}{3 \mathcal{V}^{7/3}}
           -\frac{ A e^{-a \tau}W_0  }{2 \mathcal{V}^3}
   \bigg](3\alpha_1 + 1).
  \end{aligned}
\end{equation}

\end{frame}

\subsection{KKLT à la Nilles.}
\begin{frame}{\subsecname}
\label{sec:Nilles}
En la teoría efectiva propuesta por Choi et al. en \footcite{Nilles},
el efecto $\bar D3
-$brana es puesto ``a mano'' a través de un término relacionado con la tensión
de dicho objeto. Aunque las correcciones $\alpha'$ del potencial de Kähler están
bien conocidas, hasta donde sabemos, las de las branas no han sido calculadas
todavía debido a su naturaleza no perturbativa.

Actualmente buscamos otras maneras de incluir las correcciones en el
superpotencial, o bien de calcular la tensión corregida de la $\bar D3-$brana.
\end{frame}

%%%%%%%%%%%%%%%%%%%%%%%%%%%%%%%%%%%%%%%Spurions$%%%%%%%%%%%%%%%%%%%%%%55

\section{Conclusiones.}
\begin{frame}{\secname}
\label{sec:conc}
\begin{itemize}
  \item La supersimetría es una herramienta muy útil y una teoría bastante
    interesante por sus propios méritos.
  \item Pese a lo que uno pueda oír por ahí, es indispensable tener una idea
    general de esta herramienta para poder estar al día en física teórica de
    altas energías.
  \item Aunque no haya todavía nada cierto, muchas de las avenidas que se están
    siguiendo para explicar fenómenos que todavía no entendemos (materia
    obscura, gravedad cuántica, etc) requieren de esta simetría, así que
    conviene estar atentos a lo que pasa en ese lado de la física.
\end{itemize}
\end{frame}
%%%%%%%%%%%%%%
\begin{frame}{¡Gracias por su atención!}
  \begin{center}
  \includegraphics[width=0.7\linewidth]{img.jpg}
  \end{center}
\end{frame}

\end{document}

%%%%%%%%%%BIBI$%%%%%%%%%%%
\begin{frame}[allowframebreaks]{References}
\bibliography{bibICTP}
\bibliographystyle{plain}
\end{frame}
