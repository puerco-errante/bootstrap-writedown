\documentclass[letterpaper]{article}


\usepackage{hyperref}
\usepackage{feynmp-auto}
\usepackage{tikz-feynman}
\usepackage{slashed}
\usepackage[vcentermath]{youngtab}
\usepackage{amsmath}
\usepackage{amssymb}
\usepackage{mathtools}
\usepackage[english]{babel}
\usepackage{url}
\usepackage{enumerate}
\usepackage{booktabs}
\usepackage{graphicx}
\usepackage{xcolor, colortbl}
\usepackage{longtable}
\usepackage{amsfonts}
\usepackage{amssymb}
\usepackage{caption}
\usepackage{multicol}
\usepackage{fancyhdr}
\usepackage{xspace}
\usepackage{braket}
\usepackage{rotating}
\usepackage{amsmath}
\usepackage{float}
\usepackage{listings}
\usepackage{color}
\usepackage{tikz} 
\usetikzlibrary{shapes,arrows,positioning,automata,backgrounds,calc,er,patterns}
\usepackage{tikz-feynman}
\tikzfeynmanset{compat=1.0.0}

%
\usepackage[a4paper, left=1.5cm, right=1.5cm, top=2cm, bottom=2cm]{geometry}
\setlength{\parskip}{\baselineskip}
\date{\today}

\begin{document}

\title{
  Partial Wrapup of 1808.03212.
  }
\author{
  Luviano Valenzuela Uriel Adrian
}

%%Cuántica
\newcommand{\inty}{\int_{-\infty}^{\infty}} 
\newcommand{\Lag}{\mathcal{L}} 
\newcommand{\Id}{\mathbb{I}} 
\newcommand{\Tr}{\mathrm{Tr}} 
\newcommand{\tr}{\mathrm{tr}} 
\newcommand{\Det}{\mathrm{Det}} 
\newcommand{\gauge}{\mathcal{A}} 
\newcommand{\Lac}{\ensuremath{\mathbf{\hat{L}^2}}} 
\newcommand{\La}{\ensuremath{\mathbf{\hat{L}}}} 
\newcommand{\Ham}{\ensuremath{\hat{H}}} 
\renewcommand{\a}{\ensuremath{(e_1)}}
\renewcommand{\b}{\ensuremath{(e_2)}}
\renewcommand{\c}{\ensuremath{(e_3)}}
\newcommand\cd{\mathrel{\stackrel{\makebox[0pt]{\mbox{\normalfont\tiny
CD}}}{\longleftrightarrow}}}
\newcommand{\tmop}[1]{\ensuremath{\operatorname{#1}}}
\newcommand{\then}{\ensuremath{\Rightarrow}}
\newcommand{\res}{\ensuremath{\mathrm{Res}}}
\newcommand{\R}{\ensuremath{\mathbb{R}}}
\newcommand{\integral}[1]{\ensuremath{\int_{a}^{b} #1 dt}}
%%%%%%%%%%%%%%%%%%%%
\maketitle

\section{Introduction}
Why should we study CFTs? Because they can be found in almost any area of
theoretical physics. They are crucial to describing phase transitions, appear in
several areas of particle phenomenology and lie at the very heart of String
Theory.



\section{Complex Tauberian Theorems for Laplace transform of spectral density.}

As a motivation, the paper starts by considering the four point function for
identical scalars of dimension $\Delta_\phi$, expanded as
\[
  \langle \phi(0) \phi(z,\bar z) \phi(1) \phi(\infty)\rangle
  =
  (z\bar z)^{-\Delta_\phi} \mathcal{G}(z,\bar z),
\mathcal{G}(z,\bar z)
  =
  \sum_{\Delta,J} p_{\Delta ,J} G_{\Delta, J}
  =
  \int_0^\infty dE f(E) e^{-\beta} =: \mathcal{L}(\beta),
\]
where $\beta = -(1/2)\ln u $. Using the information from the t$-$channel, one can see that for $\beta \to 0$ 
$\mathcal{L}(\beta) \to \sum_{\Delta_i} c_i \beta^{\Delta_i-2\Delta_\phi}$ and so, via the real Tauberian
theorem for the Laplace transform, the integrated density must behave as
\[
  F(E) := \int_0^E dE' f(E')\to
  \frac{E^{2\Delta_\phi}}{\Gamma(2\Delta_\phi +1)}
  (1+ (\ln E) ^{-1}).
\]
On the other hand, considering $\beta$ to run over the the complex unit circle
(which, according to the authors, amounts to taking $z$ away from the real line)
we can use the complex Tauberian Theorem which gives the following estimate for
each moment of the distribution $f(E)$:

\[
  F_m(E) := \int_0^E dE'\frac{(E-E')^{m-1} }{(m-1)!} f(E')
  \to
  E^{2\Delta_\phi}
  \left (\sum_{\Delta_i<m}
  \frac{E^{m-1 -\Delta_i}}{\Gamma(2\Delta_\phi -\Delta_i +m)}
E ^{-1} \right).
\]

It is important to note that this implies that including more operators in the t
channel, while futile for the real estimate, is highly relevant in the complex
one, as long as we consider high enough moments.

\section{Real deal}

\subsection{Harmonic Analysis trickery.}
Given the partial wave decomposition for the group $SO(d+1,1)$ we have
\begin{equation}
  \label{eq:def-pw}
  \sum_{\Delta,J} P_{\Delta,J} G_{\Delta,J}=
  \sum_{J}\int_{\frac{d}{2} }^{\frac{d}{2}
  +i\infty}\frac{d\Delta}{2\pi i} c_{J}(\Delta)F_{\Delta,J}
  =
  \sum_{J}\int_{\frac{d}{2} -i\infty}^{\frac{d}{2}
  +i\infty}\frac{d\Delta}{2\pi i} c_{J}(\Delta)K_{J,\Delta} G_{\Delta,J},
\end{equation}
where $F_{\Delta,J} = K_{J,\Delta} G_{\Delta,J}$ and we used the shadow symmetry
of $c_J$ to justify the second equality, which permits us to deform the contour
onto the real axis, picking up the poles that would naively correspond to
physical operators ($\mathrm{Res}_{\Delta \to
\Delta_0} c_J(\Delta_0) = - P_{\Delta,J}/K_{J,\Delta} $).
 However, $G_{\Delta,J}$
has poles for $\Delta$ below the unitarity bound $\Delta \geq d-2+J$, and thus
we must correct $c_J$ accordingly.
Particularly, the poles that will be relevant are of the form 
\[
  G_{\Delta,J} \sim \frac{\alpha_J(k)}{\Delta-\Delta_i} G_{J+d,J-2k}, \quad
  \Delta_i = J+ d -2k -1,\quad k = 1,2,\cdots ,[J/2].
\]
%
Demanding that \ref{eq:def-pw} be satisfied we can see that the analytic
structure of $c_J(\Delta)$ must be as follows

\[ P_{\Delta,J}/K_{J,\Delta}
=
-\mathrm{Res}_{\Delta \to
\Delta_0}
\begin {cases}
c_J(\Delta)& \Delta_0 \neq J+d + 2k-1\\
c_J(\Delta) - r_{J,\Delta} \frac{K_{\Delta + 1 -d,J+d-1}}{K_{J,\Delta}}
c_{\Delta +1-d}(J+d-1) & \Delta_0 = J+d + 2k-1 
\end{cases}
\]
where $r_{J,\Delta}$ is such that $G_{\Delta,J} - r_{\Delta +1-d,J+d-1}
G_{J+d-1,\Delta +1-d}  $ is pole free (roughly, $\res r_{J,\Delta} \sim
\alpha_J(\Delta)$). It will be of later relevance to note that the correction on
the second line depends on the large $J$ limit of $c_J(\Delta)$. 


\subsection{Inversion Formula}
Given the orthonormality of $F_{\Delta,J}$ one can project $\mathcal{G}$ onto
this base to obtain an expression for $c_J(\Delta)$. Caron-Huot proved that for
Lorentzian kinematics this formula can be written as
\begin{equation}
  c_J(\Delta)
  =
  \frac{1}{2} \delta_{J,even} \frac{\Gamma(\Delta - d/2)}{\Gamma(\Delta - 1)}
  \int_0^1 d z d\bar z \mu(z,\bar z) G_{J+d-1,\Delta+1-d}
  \mathrm{dDisc}\mathcal{G}(z, \bar z).
  \end{equation}
  By plugging in the contribution from the identity in the $t-$channel into the
  prescription for calculating dDisc we get
  \[
    \mathrm{dDisc} \left( \frac{z\bar z}{(1-z)(1-\bar z)}\right)^{\Delta_\phi}
    =2\sin^2 (\pi \Delta_\phi)\left( \frac{z\bar z}{(1-z)(1-\bar
    z)}\right)^{\Delta_\phi}
  \]
  and thence recover the generalized free theory (GFF) expression for the
  coefficient density
  \[
  c_J(\Delta)
  =
  gamma stuff
  \]
  whose large $|\delta|$ limit (via the Stirling approximation) for $0<\mathrm{arg}\Delta<\pi$ is
  \[
  c_J(\Delta)
  =
  d_J (-i\Delta)^{4\delta_\phi -3}, \quad
\delta_\phi = \Delta_\phi - \frac{d-2}{2}
  \]
  and $d_J$ is a numerical prefactor. The large $J$ limit will also prove to be
  relevant, and reads ($J$ is always even for identical scalars)
  \[
  c_J(\Delta)
  =
  b_J J^{4\delta_\phi + \frac{d}{2} -4}.
  \]
  This information sufices to write a dispersion relation for $c_J(\Delta)$
  \[
    c_J(\Delta)
    =
    \int_C \frac{d\Delta'}{2\pi i } \frac{c_J(\Delta')}{\Delta' - \Delta}.
  \]
  For $\delta_\phi <3/4$ we can drop the arcs at infinity that come from the
  usual manipulations of the controur $C$. Otherwise, we must take enough
  derivatives with respect to $\Delta '$ (substractions). Through this standard
  procedure, we can write
  \[
    c_J\left(\frac{d}{2} +  i\nu\right)
    =
    \int_0^\infty d\nu' \frac{2\nu'}{\nu'^2 + \nu^2} \rho^{OPE+extra}_J
    \left(\frac{d}{2} +  \nu'\right),
  \]
  where
  \begin{equation}
    \rho^{OPE+extra}_J \left(\frac{d}{2} +  \nu'\right) = -\sum_{poles} \delta (\Delta -
    \Delta_{pole}) \res_{\Delta \to \Delta_{pole}} c_J(\Delta)
  \end{equation}
  is defined as the collection of all the residues of $c_J$, both physical and
  kinematical. Thus, it is helpful to write

  \[
    c_J\left(\frac{d}{2} +  i\nu\right)
    +\mathrm{extra}
    =
    \int_0^\infty d\nu' \frac{2\nu'}{\nu'^2 + \nu^2} \rho^{OPE}_J
    \left(\frac{d}{2} +  \nu'\right)
  \]
  instead. In this expression one can see that the $s-$channel information is
  contained in $\rho^{OPE}$ while one could use  the $t-$channel contributions
  to calculate some limits of the $c_J$ in the LHS. $\mathrm{extra}$ will turn
  out to have both a part which can be calculated from the $t-$channel info,
  while another piece will be non universal in general.

  \subsection{How to deal with $\mathrm{extra}$.}
  Given that the form of $r_J,\Delta$ is known we can explicitly calculate the
  residues that compose $\mathrm{extra}$. They can be then written as
  \[
    \mathrm{extra} = 
    \sum_{n=odd} E_n \frac{2 \nu_n }{\nu^2 + \nu_n^2} c_{J+n+1} (J+d-1),\quad
    E_n = \frac{1}{n+1} \left[\frac{\Gamma(n/2 + 1)}{\Gamma(-n/2)\Gamma(n+1)}\right]^2
   \frac{(J+1)_{n+1}}{(J+d/2)_{n+1}}
   ,\quad
   \nu_n = J + \frac{d}{2} +n.
  \]
Now, to disentangle the contribution from each operator $\chi$ in the $t-$channel one
can rewrite
\[ E_n = 
\sum_\chi n^{\gamma_\chi -1}\sum_j \frac{e_j^{(\chi)}}{n^j} =:\sum_\chi E_n
^{\chi}       .
\]
For example, in this case $\gamma_{\hat 1} = 4\delta_\phi -2$ and
\[
e_j^{(\hat 1)}
=
2^{-\gamma_{\hat 1} +2J +d +1} \frac{\Gamma(J + d/2)\Gamma(d/2 -
\Delta_\phi)}{\Gamma(J+1)\Gamma(\Delta_\phi)}.
\]

(\emph{Is every contribution writable in this way?})

By considering this expansion, they use some clever tricks with the Lerch
trascendent and identify to types of terms: those that go as $\nu^\gamma$ and
generic  even integer powers of $\nu$.

They also consider the contribution of a primary operator $(h,\bar h)$ of
dimension $\Delta_\chi = (h+\bar h)/2$ and its
descendents to the inversion formula and denote it by
\[
  c_J^{h,\bar h} (\Delta) = 
  \nu^{4\delta_\phi - 3 -2\Delta_\chi }  \sum_{n=0}^\infty
  \frac{\alpha_n^{h,\bar h}}{\nu^{2n}}.
\]
where the $\alpha$ coefficients can be computed by considering each member in
the family and doing some excruciating manipulations of gamma functions in
integrals.

(\emph{not clear how to treat higher dim. operators in extra cont.})

They finally arrive at the following dispersion relation at large $\nu$
\[
  \int_0 ^\infty d\nu' \rho_J^{OPE} \left(\frac{d}{2} + \nu'\right)
  \frac{2\nu'\nu}{\nu'^2 + \nu^2}
  =
  \sum_\chi \sum_{n=0}^\infty \frac{\alpha_n^{\chi}}{\nu^{\delta_\chi + 2n}}+
\sum_{k=1}^\infty a_k \nu^{-2k+1}, \quad
\delta_\chi = -4\delta_\phi +2+2\Delta_\chi .
\]

The leading contribution (from the unit operator $\hat 1$) reads in this case
\[
  \alpha_0^{\hat 1} = d_J\left( 1 - \frac{1}{\cos 2\pi \delta_\phi} \right),
  \quad
  \delta_{\hat 1} = 2 -\delta_\phi.
\]

\subsection{Asymptotics and Tauberian theorems.}
Assuming that the polynomial behaviour emerges as soon as $Im[\Delta] >
|\Delta|^{\epsilon}$ for $|\Delta| >>1$ and $\epsilon$ arbitrarily small, they
use the Tauberian Theorem for the Stieltjes transform to relate the moments of
$\rho^{OPE}$ to the asymptotics derived in previous sections. Arriving at the
formula (5.3)
\[
  F_m^J(\nu) = \sum_i \alpha_i \beta_m(\delta_i)\nu^{m-\delta_i-1} +
  \sum_{k=1}^m \frac{\nu^{m-k}}{(m-k)!} + O(\nu^{\delta_{\hat 1} -1}),
\]
where one must note that the non universal integer powers (second term on the
RHS) are present up to the order of the moment considered, the succesive terms
being negligible compared to the error. Furthermore, they consider the following
linear combination
\[
  G_{m,k}^J = F^J_m + p_1 \nu F^J_{m-1} + \cdots + p_{m-k} \nu^k F^J_{m-k},
\]
where $\{p_i\}$ are chosen such that $k$ of the $b_i$ cancel and only $m-k$ such terms
are present. The general result is summarized in eq. (5.11).

\[
  G_{m,k}^J = \sum_i\left(\frac{\cos \pi\delta_i/2}{\pi} \frac{\Gamma(1-m)
  \Gamma(k-\delta_i)}{\Gamma(1+k-m) \Gamma(m-\delta_i)}\right) \alpha_i
  \nu^{m-1-\delta_i} + b_{k+1} \nu^{m-k-1} +\cdots + b_m +
  O(\nu^{k-\delta_{\hat 1} -1}).
\]

\subsection{Examples}
For GFF in 3d, the authors reproduce the results from the literature with an
incredible precision. For the 2d Ising model, the agreement is quite good. In
the case of the 3d Ising model, I couldn't understand quite well what their
predictions are.

\section{Questions?}
\begin{itemize}
  \item Every time a coefficient has a Gamma-function pole, the authors choose
    the prescription of discarding it and keeping the log term. Is this
    conventional? Justifiable?
%
  \item How pursuable is to check the assumed behaviour near the real axis of
    $\Delta$?
\end{itemize}


\section{Filling in the gaps.}

\subsection{Tauberian theorem for the Stieltjes transform and its application to
dispersion relations for the OPE.}
The Tauberian Theorem for the Stieltjes states that given two positive, even
functions $\rho(\nu)$, $\phi(\nu)$ whose Stieltjes transforms are related by
\begin{equation}
  \int_\infty^\infty d\nu \frac{\rho(\nu)}{\nu - z}
  -
  \int_\infty^\infty d\nu \frac{\phi(\nu)}{\nu - z}
  =
  R(z),
\end{equation}
with $R(z)$ analytic for $z\in G$ and $G = \{ z = x+ iy | |y|\geq\Lambda(x)\}$
with $\Lambda(x)$ a sufficiently smooth function and $R(z) = O(|z|^{\omega})$
($\omega$ as defined below the next equation)
when $z\to \infty$, satisfy the following relation ($y:=\Lambda(x)$)

\begin{equation}
  F_m(x)
  :=
  \int_\infty^\infty \frac{dx (x-\nu)}{(m-1)!} \rho(x)
  =
  \Phi_m(x)
  +
  \sum_{k=1}^m \frac{ b^k x^{m-k}}{(m-k)!},
  +O(y^m (x^{m-\omega} + \phi(x)) + x^{-1} y^{m+1}|\phi(-x)|)
\end{equation}
where the $b_k$ are defined in terms of $R(z)$, which must have an asymptotic
behaviour satisfying $\omega >m$.

This is done by integrating on a clever contour to approximate the Stieltjes
Kernel by a step function and then approximating the remainder.

To match with the situation discussed in the paper, they have to correct some
approximations for the case where $\rho(\nu)$ is uneven (since $\rho \sim \delta
(\nu) c_J(\nu)$), which 






















  \end{document}
